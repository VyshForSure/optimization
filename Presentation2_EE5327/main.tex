\documentclass{beamer}
\usepackage{tikz}
\newcommand*\circled[1]{\tikz[baseline=(char.base)]{% <---- BEWARE
            \node[shape=circle,draw,inner sep=2pt] (char) {#1};}}


%
% Choose how your presentation looks.
%
% For more themes, color themes and font themes, see:
% http://deic.uab.es/~iblanes/beamer_gallery/index_by_theme.html
%
\mode<presentation>
{
  \usetheme{default}      % or try Darmstadt, Madrid, Warsaw, ...
  \usecolortheme{default} % or try albatross, beaver, crane, ...
  \usefonttheme{default}  % or try serif, structurebold, ...
  \setbeamertemplate{navigation symbols}{}
  \setbeamertemplate{caption}[numbered]
} 

\usepackage[english]{babel}
\usepackage[utf8x]{inputenc}

\title[Your Short Title]{Simplex Presentation - EE5327}
\author{Srujana B - MA17BTECH11001\\J Sai Vyshnavi - MA17BTECH11005}
\date{28/02/2019}

\begin{document}

\begin{frame}
  \titlepage
\end{frame}

% Uncomment these lines for an automatically generated outline.
%\begin{frame}{Outline}
%  \tableofcontents
%\end{frame}

\section{Question}

\begin{frame}{Question}

\begin{itemize}
  \item Consider the linear programming problem\\
  Maximize $3x + 9y$,\\
  Subject to\\
        \hspace{2 cm}$2y-x<=2$,\\
        \hspace{2 cm}$3y-x>=2$,\\
        \hspace{2 cm}$2x+3y<=10$,\\
        \hspace{2 cm}$x,y>=0$.\\
Then the maximum value of the objective function is 
\end{itemize}

\vskip 1cm

\end{frame}

\begin{frame}{Solution using CVXPY}

import cvxpy as cvx\\
from numpy import matrix, round, eye\\
import numpy\\
X = cvx.Variable()\\
Y = cvx.Variable()\\

constraints = [$-X+2*Y<=2$,\\
\hspace{2.3 cm}$3*Y-X>=0$,\\
\hspace{2.3 cm}$2*X+3*Y<=10$,\\
\hspace{2.3 cm} $X>=0$,\\
\hspace{2.3 cm}$Y>=0$]\\

obj = cvx.Maximize(3*X+9*Y)\\

prob = cvx.Problem(obj,constraints)\\
prob.solve()\\

print("Status:", prob.status)\\
print("Optimal value", prob.value)\\
print("Optimal var", X.value, Y.value)\\

\end{frame}

\begin{frame}{Solution using CVXPY}
\begin{itemize}
  \item Status: optimal\\
Optimal value 24.00\\
Optimal var 2.00 2.00\\
\end{itemize}

\vskip 1cm
\end{frame}

\begin{frame}{Solution using SIMPLEX Method(Tabular)}
\textbf{Step 1} : 
Introduce Slack variables $\mu_{1},\mu_{2},\mu_{3}$.\\
\textbf{Step 2} : Transform the given LPP\\
After transformation given problem turns into :\\
 Maximize $3x + 9y+0\mu_{1}+0\mu_{2}+0\mu_{3}$,\\
  Subject to\\
  \hspace{2 cm}$2y-x+\mu_{1}=2$,\\
        \hspace{2 cm}$x-3y+\mu_{2}=2$,\\
        \hspace{2 cm}$2x+3y+\mu_{3}=10$,\\
        \hspace{2 cm}$x,y,\mu_{1},\mu_{2},\mu_{3}>=0$.\\

\textbf{Step 3} : Construct the tabulae
\end{frame}


\begin{frame}{Solution using SIMPLEX Method(Tabular)}

\begin{center}
\begin{tabular}{ c|ccccc|c|c| } 
  $c_{i}$&3 & 9 & 0 & 0 & 0 &  &  \\ 
  &x &$\rightarrow$\textbf{\textit{y}}& $\mu_{1}$ & $\mu_{2}$ & $\mu_{3}$ & RHS &
  $\theta$\\
  \hline
$\rightarrow$\textbf{0$\mu_{1}$} &-1 & \textbf{2} & 1 & 0 & 0 & 2 &\circled{1} \\ 0$\mu_{2}$ &1 & -3 & 0 & 1 & 0 & 0  & - \\ 
 0$\mu_{3}$ &2 & 3 & 0 & 0 & 1 & 10 & 5 \\ 
 \hline
  $c_{i}-z_{i}$&3 & \circled{9} & 0 & 0 & 0 & $Z_{RHS}=0$&  \\ 
  \hline
\end{tabular}
\vskip  1cm

\end{center}

   Swap the variables $ \mu_{1}$ and y\\
   Here the pivot element is 2,\\
   Make it 1 and other elements in its column 0 using elementary row transformations.\\
   
  \hspace{1cm} $R_{1}= R_{1}/2$\\
  \hspace{1cm}  $R_{2}= R_{2}+3R_{1}$\\
   
   \hspace{1cm} $R_{3}=R_{3}-3R_{1}$\\
   
\end{frame}

\begin{frame}{Solution using SIMPLEX Method(Tabular)}

\begin{center}
\begin{tabular}{ c|ccccc|c|c| } 
  $c_{i}$&3 & 9 & 0 & 0 & 0 &  &  \\ 
  &$\rightarrow$\textbf{\textit{x}} &y& $\mu_{1}$ & $\mu_{2}$ & $\mu_{3}$ & RHS &
  $\theta$\\
  \hline
 0$\mu_{1}$ &-1 & \circled{\textbf{2}} & 1 & 0 & 0 & 2 &\circled{1} \\ 0$\mu_{2}$ &1 & -3 & 0 & 1 & 0 & 0  & - \\ 
 0$\mu_{3}$ &2 & 3 & 0 & 0 & 1 & 10 & 5 \\ 
 \hline
  $c_{i}-z_{i}$&3 & \circled{9} & 0 & 0 & 0 & $Z_{RHS}=0$&  \\ 
  \hline
  9y &-1/2 & 1 & 1/2 & 0 & 0 & 1 &- \\ 
  0$\mu_{2}$ &-1/2 & 0 & 3/2 & 1 & 0 & 3  & - \\ 
 $\rightarrow$\textbf{0$\mu_{3}$} &\circled{\textbf{7/2}} & 0 & -3/2 & 0 & 1 & 7 & \circled{2} \\ 
 \hline
  $c_{i}-z_{i}$&\circled{15/2} & 0 & -9/2 & 0 & 0 & $Z_{RHS}=9$&  \\ 
  \hline
\end{tabular}
\end{center}
   Swap the variables $ \mu_{3}$ and x\\
   Here the pivot element is 7/2,\\
   Make it 1 and other elements in its column 0 using elementary row transformations.\\
   
\end{frame}


\begin{frame}{Solution using SIMPLEX Method(Tabular)}

\begin{center}
\begin{tabular}{ c|ccccc|c|c| } 
  $c_{i}$&3 & 9 & 0 & 0 & 0 &  &  \\ 
  &$\rightarrow$\textbf{\textit{x}} &y& $\mu_{1}$ & $\mu_{2}$ & $\mu_{3}$ & RHS &
  $\theta$\\
  \hline
 0$\mu_{1}$ &-1 & \circled{\textbf{2}} & 1 & 0 & 0 & 2 &\circled{1} \\ 0$\mu_{2}$ &1 & -3 & 0 & 1 & 0 & 0  & - \\ 
 0$\mu_{3}$ &2 & 3 & 0 & 0 & 1 & 10 & 5 \\ 
 \hline
  $c_{i}-z_{i}$&3 & \circled{9} & 0 & 0 & 0 & $Z_{RHS}=0$&  \\ 
  \hline
  9y &-1/2 & 1 & 1/2 & 0 & 0 & 1 &- \\ 
  0$\mu_{2}$ &-1/2 & 0 & 3/2 & 1 & 0 & 3  & - \\ 
 $\rightarrow$\textbf{0$\mu_{3}$} &\circled{\textbf{7/2}} & 0 & -3/2 & 0 & 1 & 7 & \circled{2} \\ 
 \hline
  $c_{i}-z_{i}$&\circled{15/2} & 0 & -9/2 & 0 & 0 & $Z_{RHS}=9$&  \\ 
  \hline
 9y &0 & 1 & 2/7 & 0 & 1/7 & \textbf{2}$\leftarrow$ & \\
 0$\mu_{2}$ &0 & 0 & 9/7 & 1 & 1/7 & 4  & \\ 
 3x &1 & 0 & -3/7 & 0 & 2/7 & \textbf{2}$\leftarrow$ &  \\ 
 \hline
  $c_{i}-z_{i}$&0 & 0 & -12/7 & 0 & -15/7 & $Z_{RHS}=24$&  \\ 
  \hline
\end{tabular}
\end{center}
\end{frame}

\begin{frame}{Solution}

Since the values of $c_{i}-z_{i}$ are all non positive, we can end the iterations.\\
Thus, the optimal values are as follows \\
\hspace{1cm} x = 2\\
\hspace{1cm} y = 2\\
 Objective Function = 3x+9y = 24.
\end{frame}

\end{document}
